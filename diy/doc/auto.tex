\cutname{auto.html}

The authors of~\dont{} are Jade Alglave and Luc Maranget
(INRIA Paris--Rocquencourt).

\section{Preamble}
Following Part~\ref{part:diy}, we describe our tests \emph{via} cycles, built
from the candidate relaxations they involve.  We consider a candidate
relaxation to be \emph{relaxed}, or \emph{non-global}, when it corresponds to
the weaknesses that can be observed on a system implementing $A$. We consider a
candidate relaxation to be \emph{safe}, or \emph{global}, when it is
guaranteed, $\eg$ by the documentation, never to be relaxed.

In the following, we consider an architecture $A$ to be a pair $(\relaxs_{A},
\safes_{A})$, where $\relaxs_{A}$ (resp.  $\safes_{A}$) are the candidate
relaxations relaxed (resp. safe) for $A$.
The automated front-end \dont{} mechanises the task of checking
that a machine or executable model conforms to such an architecture, and of
exploring architectures.
We provide some experiment reports
\footahref{http://diy.inria.fr/dont/dont/index.html}{elsewhere}.
This document is intended to be a gentle introduction to~\dont{} and
a partial reference.



\section{A tour of~\dont}

\subsection{Checking \aname{conform}{conformance}}
We want to check that a given machine $M$ is conform to an
architecture $A$. By conform, we mean that the machine $M$ does not exhibit
more behaviours than the architecture $A$ actually allows.

For example, let us consider an x86 machine with $2$~processors. Suppose that
we have been told that x86 machines are TSO~\cite{sparcboth}, and that we want
to check that.
As the default values of~\dont{} options handle that very situation, we type:
\begin{verbatim}
$ dont -mode conform
** Step 0 **
Phase 2 in A (6 tests)
...
Phase 2 in A (6 tests)
** Step 5 **
Safe set {Rfe, Fre, Wse, PodWW, PodRW, PodRR, MFencedWR} is conform
\end{verbatim}
The automated front-end \dont,
assumed the TSO safe set (the default for x86),
called the \diy{} tool (see Part~\ref{part:diy}) to generate
all the tests that are forbidden by TSO --- up to $2$~processors;
ran them ($5$ times) with our companion \prog{litmus} tool, (see
Part~\ref{part:litmus}) against our x86 machine;
and observed that the machine does not exhibit any outcome
forbidden by TSO.
In effect, \dont{} in conformance check mode automates the safe
tests of Sec.~\ref{safe:test:sec}. 


\subsection{Checking  non-conformance}
Now, we wish to prove that an x86 machine is not
sequentially consistent.
To that end, we write the following configuration file~\afile{x86.sc}:
\verbatiminput{x86.sc}
Most of \dont{} controls are set, sometimes to
their default values:
\begin{itemize}
\item \opt{arch = X86} sets the targeted architecture,
\opt{mode = conform} sets conformance check mode,
and \opt{stablise = 1} commands performing the check round once
(the default is five times, cf. \emph{supra}).
\item \opt{safe = Rfe,Fre,Wse,Pod**,[Rfi,PodRR]} defines
the set of safe relaxation candidates used to generate litmus tests
(up to $2$ processors, by \opt{nprocs = 2}).
\item The front-end~\dont{} calls \litmus{} and runs the tests
with the specified options.
The setting \opt{litmus\_opt = -a 2 -i 0} specifies that two processors
are available and enables affinity control (see Sec.~\ref{litmus:option:sec}
for the description of \litmus{} options).
Tests will be run twice per check round, once
with options \opt{-s 100000 -r 10},
and once with options \opt{-s 5000 -r 200 -i 1}
(see Sec.~\ref{exec:control} for the description of test executable options).
Finally, the setting \opt{build = make -j 2 -s} specifies the command to use
to compile the C~source files that \litmus{} produces.
\end{itemize}
We run \dont{} configured by \afile{x86.sc} as follows:
\begin{verbatim}
$ dont x86.sc 
** Step 0 **
Phase 2 in A (9 tests)
...
** Step 1 **
Safe set {[Rfi,PodRR], Rfe, Fre, Wse, PodWW, PodWR, PodRW, PodRR} is not conform
++ Invalidating tests ++
A006: 'Fre PodWR Fre PodWR' {Fre, PodWR}
A007: 'Fre PodWW Wse PodWR' {Fre, Wse, PodWW, PodWR}
A001: 'Rfi PodRR Fre PodWR Fre' {[Rfi,PodRR], Fre, PodWR}
A002: 'Rfi PodRR Fre PodWW Wse' {[Rfi,PodRR], Fre, Wse, PodWW}
A000: 'Rfi PodRR Fre Rfi PodRR Fre' {[Rfi,PodRR], Fre}
++++++++
\end{verbatim}
The conformance check failed and the tests that invalidate the hypothesis
``x86 is sequentially consistent'' are listed.
The check took place in directory~\opt{A}.
Directory~\opt{A} contains the actual logs of \litmus{} runs
as files \opt{A.00}, \opt{A.01} etc.,
in addition to the sources of the litmus tests:
\begin{verbatim}
$cat A/A006.litmus
X86 A006
"Fre PodWR Fre PodWR"
Cycle=Fre PodWR Fre PodWR
Relax=
Safe=Fre PodWR
{ }
 P0          | P1          ;
 MOV [x],$1  | MOV [y],$1  ;
 MOV EAX,[y] | MOV EAX,[x] ;
exists (0:EAX=0 /\ 1:EAX=0)
\end{verbatim}
Notice that, since tests are described by their cycles,
the source of tests can also be reconstructed with~\ahrefloc{diyone}{\diyone}:
\begin{verbatim}
% diyone -arch X86 Fre PodWR Fre PodWR
X86 a
"Fre PodWR Fre PodWR"
{ }
 P0          | P1          ;
 MOV [y],$1  | MOV [x],$1  ;
 MOV EAX,[x] | MOV EAX,[y] ;
exists (0:EAX=0 /\ 1:EAX=0)
\end{verbatim}

\subsection{Automatically exploring the memory model exhibited by a  machine}
Now suppose that we have no idea of the memory model of
our $2$ processors x86 machine.
Another mode of our \dont{} tool automatically explores a given
machine, and outputs an architecture ($\ie$ a pair $(\relaxs_{A},
\safes_{A})$) to which the machine conforms.
The following configuration file~\afile{x86.explo}
instructs~\dont{} to perform such an exploration.
\verbatiminput{x86.explo}
With respect to conformance check, new or changed settings are the selection
of exploration mode by \opt{mode = explo},
the definition of the initial safe set by \opt{safe = Fre,Wse},
and and the definition of the candidate relaxations to be tested
(\opt{testing = Rfe,Pod**,MFenced**,[Rfi,PodR*]}).


We launch the exploration as:
\begin{verbatim}
$ dont x86.explo
\end{verbatim}
The whole process only takes a few minutes, mostly
due to the limited number of tests induced by the setting \opt{nprocs = 2}.

We now detail \dont{} output
\ifhevea (complete log of the exploration: \afile{x86-log.txt})
\else (the \footahref{\url{http://diy.inria.fr/doc/auto.html}}{html version}
of this document includes the complete log of the experience)\fi.
We start by a first exploration round:
\begin{verbatim}
** Step 0 **
Testing: {[Rfi,PodRW], [Rfi,PodRR], Rfe, PodWW, PodWR, PodRW, PodRR, MFencedWW, 
MFencedWR, MFencedRW, MFencedRR}
Relaxed: {}
Safe   : {Fre, Wse}
Phase 1 in A (6 tests)
Actually tested: {[Rfi,PodRW], [Rfi,PodRR], PodWW, PodWR, MFencedWW, MFencedWR}
Added relax: {[Rfi,PodRR], PodWR}
Added safe: {[Rfi,PodRW], PodWW, MFencedWW, MFencedWR}
Phase 2 in B (6 tests)
\end{verbatim}
The log above first indicates the current status of exploration
as three sets: \emph{testing}, \emph{relaxed} and~\emph{safe}.
Initially, no candidate relaxation has yet been observed to be relaxed,
while the testing and safe sets are as assumed.
Each exploration round is divided in two phases.
The aim of Phase~1 (performed in directory~\opt{A})
is to classify some candidate relaxations as either relaxed or safe.
It here
succeeds for 6 candidate relaxations, whose observed status is indicated.
Phase~2 (performed in directory~\opt{B})
basically is a conformance check of the current safe set.
The conformance check succeeds and all safe candidate relaxations
found at phase~1 make it to the next round:
\begin{verbatim}
** Step 1 **
Testing: {Rfe, PodRW, PodRR, MFencedRW, MFencedRR}
Relaxed: {[Rfi,PodRR], PodWR}
Safe   : {[Rfi,PodRW], Fre, Wse, PodWW, MFencedWW, MFencedWR}
Phase 1 in C (10 tests)
Actually tested: {Rfe, PodRW, PodRR, MFencedRW, MFencedRR}
Added safe: {Rfe, PodRW, PodRR, MFencedRW, MFencedRR}
Phase 2 in D (17 tests)
\end{verbatim}
Phase~1 (performed in directory~\opt{C})
can now target new candidate relaxations, because of the increased
safe set. All of targeted candidate relaxations
are observed to be safe, which is confirmed
by phase~2. As a  consequence, there does not remain
any candidate relaxation to be tested and the next round reduces
to a conformance check:
\begin{verbatim}
** Step 2 **
Testing: {}
Relaxed: {[Rfi,PodRR], PodWR}
Safe   : {[Rfi,PodRW], Rfe, Fre, Wse, PodWW, PodRW, PodRR, MFencedWW, MFencedWR, MFencedRW, MFencedRR}
Phase 1 in E (0 tests)
Phase 2 in D (17 tests)
\end{verbatim}
The same check is performed
for $4$~additional rounds as governed by the default
value of $5$ for the setting of~\opt{stabilise}.
Round number~$6$ then shows the result of exploration,
(\emph{i.e.} the pair $(\relaxs_{A}, \safes_{A})$), prefixed
by the list of tests that justify observed relaxations:
\begin{verbatim}
** Step 6 **
...
++ Witness(es) for relaxed [Rfi,PodRR] ++
A001: 'Rfi PodRR Fre Rfi PodRR Fre' {[Rfi,PodRR], Fre}
++++++++
++ Witness(es) for relaxed PodWR ++
A003: 'Fre PodWR Fre PodWR' {Fre, PodWR}
++++++++
Observed relaxed: {Rfi, PodWR}
Observed safe: {Rfe, Fre, Wse, PodWW, PodRW, PodRR, MFencedWW, MFencedWR, MFencedRW, MFencedRR}
\end{verbatim}
And we go again for $5$~additional rounds of pure conformance check:
\begin{verbatim}
** Now checking safe set conformance **
** Step 7 **
Phase 2 in F (17 tests)
...
* Step 12 **
Observed relaxed: {Rfi, PodWR}
Observed safe: {Rfe, Fre, Wse, PodWW, PodRW, PodRR, MFencedWW, MFencedWR, MFencedRW, MFencedRR}
\end{verbatim}


Once exploration is complete, all litmus tests and logs of \litmus{} runs are
still present in their directories \opt{A}, \opt{B}, etc.
For instance, the directory~\opt{F} contain the $10$ logs of the final
conformance check, as the files \opt{F.01}, \ldots, \opt{F.09}:
\begin{verbatim}
$ ls F/F.??
F/F.00  F/F.01  F/F.02  F/F.03  F/F.04  F/F.05  F/F.06  F/F.07  F/F.08  F/F.09
\end{verbatim}

The tool \dont{} offers a convenient replay feature:
\begin{verbatim}
$ dont -restart
** Step 0 *
...
* Step 12 **
Observed relaxed: {Rfi, PodWR}
Observed safe: {Rfe, Fre, Wse, PodWW, PodRW, PodRR, MFencedWW, MFencedWR, MFencedRW, MFencedRR}
\end{verbatim}
The command above takes a few seconds of time, since experiments
are not run again. Instead, the logs of \litmus{} runs are read and
their interpretation is re-performed.
Notice that the restart feature also permits to pursue interrupted
experiments.

\section{Usage of~\dont}
In effect, the tool~\dont{} automates the complete testing
procedure described in the documentation of \diy{} proper
(Sec.~\ref{diy:intro}).
It is to be noticed that \dont{} requires a fully functional installation
of the \diy~tool suite.
In particular, the commands~\diy{} and~\litmus{} must be installed
and runnable as ``\diy'' and~``\litmus'' (\emph{i.e.} installed in path).



\subsection{Command-line options}
The automated front-end \dont{} is configured mostly by the means
of a configuration file, which \dont{} takes as a command-line argument.
Nevertheless, \dont{} accepts the following, limited, set of options:
\begin{description}
\item[{\tt -v}] Be verbose, repeat to increase verbosity.
\item[{\tt -version}]  Show version number and exit.
\item[{\tt -arch (X86|PPC|ARM)}] Set architecture. Default is~\opt{X86}.
ARM is untested.
\item[{\tt -mode (conform|explo)}] Set main mode,
either conformance check or exploration. Default is~\opt{explo}.
\item[{\tt -nprocs <n>}]
Generate tests up to <n> processors (defaults: X86=2, PPC=4)
\item[{\tt -restart}] Restart the experiment in hand in current directory.
\end{description}
Except for \opt{-restart} command lines options are not intended
for normal use.
In particular, command-line options do not override values defined
in configuration files.

Namely, there are many parameters to set and appropriate values
for them will depend on the tested machine.
In particular, \litmus{} parameters need to be chosen carefully, by
the means of preliminary experiments.
For instructions on configuring~\litmus{}, refer to
Sec.~\ref{litmus:control} of \litmus{} documentation.


\subsection{Configuration files}
The general syntax of configurations files is a sequence of
lines \textit{key}\texttt{ = }\textit{value}.
Comment lines are introduced by \verb+#+.
The tool \dont{} recognises the following keys:

\paragraph*{General behaviour}
\begin{description}
\item[{\tt mode = (conform|explo)}]
Main operating mode. Default is~\opt{explo}
\item[{\tt arch = (X86|PPC|ARM)}]
Target architecture. Default is~\opt{X86}.
\item[{\tt run = (local|ssh <addr>|cross <addr1> <addr2>)}]
Give access to the tested machine,
which can be either the machine where \dont{} runs,
or remote machine \opt{<addr>},
or compile C~files on remote machine \opt{addr1} and execute on
tests on remote machine \opt{addr2}.
Machine addresses are \opt{[user@]machine[:port]} expressing
connection elements for both \texttt{ssh} and~\texttt{scp}.
Default is~\opt{local}.

\item[{\tt work\_dir = \textit{dir}}]
Directory for temporary files, default is \opt{/var/tmp}.
\item[{\tt stabilise = <\textit{n}>}]
In conformance check mode, \dont{} performs $n$ rounds of conformance testing.
In exploration mode, \dont{} ends the exploration after $n$~rounds
without state change. Default is~$5$.
\item[{\tt interactive = <bool>}]
In exploration mode and after $n$~rounds
without state change, \dont{} will either assume that the whole current
testing set is safe (\opt{false}), or ask the user (\opt{true}) to
decide for some of the elements of this set to be safe.
Default is~\opt{true}, \emph{i.e.} ask user.
\end{description}

\paragraph{Controlling Cycle Generation}
\begin{description}
\item[{\tt nprocs = <\textit{n}>}]
Generate cycles up to $n$ processors. Default is $2$ for~x86 and~$4$ for Power.
\item[{\tt diy\_sz = <\textit{m}>}]
Upper limit on the size of cycles of candidate relaxations.
Default is $2 \times n$, where $n$ is the number of processors.
With decent values of the initial candidate relaxations sets
(see below), this default commands the generation of all
(critical, see~Sec.~\ref{critical:def}) cycles that involve
up to $n$~processors.
\item[{\tt safe = <\textit{relax-list}>}]
Define the safe set~$S$. In exploration mode, $S$ is the initial value
of the safe set (default \opt{Fre, Wse}). In conformance
mode, $S$ is the safe set checked.
Ddefault is \opt{Rfe, Fre, Wse, PodR*, PodWW, MFencedWR} for x86,
and unspecified for other architectures.
\item[{\tt testing = <\textit{relax-list}>}]
Define the tested set of candidate relaxations.
The tested set is relevant only in exploration mode.
Default values are
\opt{Rfe,Pod**,MFenced**,[Rfi,MFencedR*],[Rfi,PodR*]} for x86
and unspecified for other architectures.
\end{description}
The syntax for \textit{relax-list} above is a comma (or space) separated
list of candidate relaxations.
Candidate relaxations are introduced by the documentation of~\diy{}
(see Part~\ref{part:diy})

\paragraph{Control of external tools}
\begin{description}
\item[{\tt litmus\_opts = <\textit{opts}>}]
Define options used by~\dont{} when it calls \litmus.
Default is the empty string, \emph{i.e.} use \litmus{} defaults.
\item[{\tt run\_opts = <\textit{opts$_1$},\ldots,\textit{opts$_n$}>}]
Define options used for running litmus tests.
Any set of litmus tests generated and compiled by~\dont,
will be run $n$ times, with specified options.
More concretely, \dont{} will run the litmus tests with commands
\texttt{sh run.sh \textit{opts$_1$}}, \ldots,
\texttt{sh run.sh \textit{opts$_n$}}.
The default is the empty string, \emph{i.e.} run tests once with no option.
\item[{\tt build = <\textit{command}>}]
Defines the command issued by~\dont{} to compile the C~source files
produced by \litmus.
The default is \opt{sh comp.sh}, \emph{i.e.} runs the compilation script
produced by \litmus. An interesting alternative is
\texttt{make -s -j \textit{n}} for concurrent compilation,
with up to~$n$ concurrent tasks.
\end{description}

\endinput
We now perform a TSO-conformance check of our machine \opt{saumur}.
The machine \opt{saumur} is an  $8$~(logical) processors Intel Xeon.
However,
we shall limit conformance tests to those that involve up to $4$~processors
--- generating tests up to $8$~processors is doable, but the conformance
check would then last several days.
We write the following configuration file~\afile{saumur.tso}:
\verbatiminput{saumur.tso}
The configuration file above for~\dont{}
first specifies conformance check mode,
instructs \diy{} to generate cycles up to $4$ processors,
then instructs \dont{} to call \litmus{} with the given options.
Again, the there will be $5$~runs of tests.
However (\opt{run\_opts = -i 1,-i 4}), each of these five runs will consist
in running the tests twice, first with option \opt{-i 1} and then
with option \opt{-i 4}. Finally tests are compiled with the specified build
command.
We run \dont{} as follows:
\begin{verbatim}
$ dont saumur.tso
** Step 0 **
Phase 2 in A (68 tests)
...
** Step 5 **
Safe set {Rfe, Fre, Wse, PodWW, PodRW, PodRR, MFencedWR} is conform
\end{verbatim}
The conformance check lasts for about one hour and succeeds.
