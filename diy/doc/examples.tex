\cutname{examples.html}
In the following experiment reports we describe both how we generate tests
and how we run them on various machines under various conditions.

\section{Running several tests at once, changing critical parameters}
In this section we describe an experiment on changing the stride
(cf Sec.~\ref{defstride}).
This usage pattern applies to many situations, where a series of
test is compiled once and run many times under changing conditions.

We assume a directory \ahref{tst-x86}{\texttt{tst-x86}}%
\ifhevea{} (\ahref{tst-x86.tar}{archive})\fi,
that contains a series of litmus tests
and an index file~\ahref{tst-x86/@all}{\file{@all}}.
Those tests where produced by the \prog{diy} tool (see Sec.~\ref{diy:intro}).
They are two thread tests that exercise various relaxed behaviour
of x86~machines.
More specifically, \diy{} is run as ``\texttt{diy -conf X.conf}'',
where \ahref{tst-x86/X.conf}{\texttt{X.conf}} is the
following configuration file
\verbatiminput{tst-x86/X.conf}
As described in Sec.~\ref{diy:usage}, \diy{} will generate all
\emph{critical} cycles of size at most 5, built from the given lists
of candidate relaxations, spanning other two threads,
and including at least one occurrence of PodWR, [Rfi,PodRR] or both.
In effect, as x86 machines follow the TSO model that relaxes write to read
pairs, all produced tests should \emph{a priori} validate.


We test some \texttt{x86-64} machine, using the following 
\ahref{x86-64.cfg}{\texttt{x86-64.cfg}} \litmus{} configuration file:
\verbatiminput{x86-64.cfg}
The number of available logical processors is unspecified,
it thus defaults to~$1$,
leading to running one instance of the test only (cf parameter $a$ in Sec.~\ref{defa})

We invoke \litmus{} as follows,
where \texttt{run} is a pre-existing empty directory:
\begin{verbatim}
% litmus -mach x86-64 -o run tst-x86/@all
\end{verbatim}
The directory \texttt{run} now contains C-source files for the tests,
as well as some additional files:
\begin{verbatim}
% ls run
comp.sh   outs.c  README.txt  utils.c  X000.c  X002.c  X004.c  X006.c
Makefile  outs.h  run.sh      utils.h  X001.c  X003.c  X005.c
\end{verbatim}
One notices a short \texttt{README.txt} file, two scripts to compile
(\texttt{com.sh}) and run the tests (\texttt{run.sh}), and a \texttt{Makefile}.
We use the latter to build test executables:
\begin{verbatim}
% cd run
% make -j 8
gcc -Wall -std=gnu99 -fomit-frame-pointer -O2 -m64 -pthread -O2 -c outs.c
gcc -Wall -std=gnu99 -fomit-frame-pointer -O2 -m64 -pthread -O2 -c utils.c
gcc -Wall -std=gnu99 -fomit-frame-pointer -O2 -m64 -pthread -S X000.c
...
gcc -Wall -std=gnu99 -fomit-frame-pointer -O2 -m64 -pthread  -o X005.exe outs.o utils.o X005.s
gcc -Wall -std=gnu99 -fomit-frame-pointer -O2 -m64 -pthread  -o X006.exe outs.o utils.o X006.s
rm X005.s X004.s X006.s X000.s X001.s X002.s X003.s
\end{verbatim}
This builds the seven tests \texttt{X000.exe} to~\texttt{X006.exe}.
The size parameters (\verb+size_of_test = 1000+ and
\verb+number_of_run = 10+) are rather small, leading to fast tests:
\begin{verbatim}
% ./X000.exe 
Test X000 Allowed
Histogram (2 states)
5000  :>0:EAX=1; 0:EBX=1; 1:EAX=1; 1:EBX=0;
5000  :>0:EAX=1; 0:EBX=0; 1:EAX=1; 1:EBX=1;
No
...
Condition exists (0:EAX=1 /\ 0:EBX=0 /\ 1:EAX=1 /\ 1:EBX=0) is NOT validated
...
Observation X000 Never 0 10000
Time X000 0.01
\end{verbatim}
However, the test fails, in the sense that the relaxed outcome targeted by
\texttt{X000.exe} is not observed, as can be seen quite easily from
the ``\texttt{Observation Never\ldots}'' line above .


To observe the relaxed outcome,
it happens it suffices to change the stride value to~$2$:
\begin{verbatim}
% ./X000.exe -st 2
Test X000 Allowed
Histogram (3 states)
21    *>0:EAX=1; 0:EBX=0; 1:EAX=1; 1:EBX=0;
4996  :>0:EAX=1; 0:EBX=1; 1:EAX=1; 1:EBX=0;
4983  :>0:EAX=1; 0:EBX=0; 1:EAX=1; 1:EBX=1;
Ok
...
Condition exists (0:EAX=1 /\ 0:EBX=0 /\ 1:EAX=1 /\ 1:EBX=0) is validated
...
Observation X000 Sometimes 21 9979
Time X000 0.00
\end{verbatim}

We easily perform a more complete experiment with the stride changing
from~$1$ to~$8$, by running the \texttt{run.sh} script,
which transmits its command line options to all test executables:
\begin{verbatim}
% for i in $(seq 1 8)
> do
> sh run.sh -st $i > X.0$i
> done
\end{verbatim}
Run logs are thus saved into files \texttt{X.01} to~\texttt{X.08}.
The following table summarises the results:
\begin{center}\let\handletest\xhandletest
\begin{tabular}{|l||r|r|r|r|r|r|r|r|}
\hline
& X.01 & X.02   & X.03 & X.04  & X.05 & X.06  & X.07 & X.08   \\
\hline
\hline
\handletest{X000} &  0/10k& 21/10k & 0/10k& 17/10k& 0/10k& 19/10k& 2/10k& 40/10k \\
\hline
\handletest{X001} &  0/10k& 108/10k& 0/10k& 77/10k& 2/10k& 29/10k& 0/10k& 29/10k \\
\hline
\handletest{X002} &  0/10k& 2/10k  & 0/10k& 6/10k & 0/10k& 7/10k & 0/10k& 5/10k  \\
\hline
\handletest{X003} &  0/10k& 4/10k  & 2/10k& 1/10k & 0/10k& 5/10k & 0/10k& 11/10k \\
\hline
\handletest{X004} &  0/10k& 4/10k  & 0/10k& 33/10k& 0/10k& 10/10k& 0/10k& 8/10k  \\
\hline
\handletest{X005} &  0/10k& 1/10k  & 0/10k& 0/10k & 0/10k& 5/10k & 0/10k& 4/10k  \\
\hline
\handletest{X006} &  0/10k& 8/10k  & 0/10k& 9/10k & 0/10k& 11/10k& 1/10k& 12/10k \\
\hline
\end{tabular}


\end{center}
For every test and stride value cells show how many times the targeted relaxed
outcome was observed/total number of outcomes.
One sees that even stride value perfom better --- noticeably $2$, $6$ and~$8$.
Moreover variation of the stride parameters permits the observation of
the relaxed outcomes targeted by all tests.


We can perform another, similar, experiment changing the $s$ (\verb+size_of_test+) and~$r$ (\verb+number_of_run+) parameters.
Notice that the respective default values of $s$ and~$r$ are
$1000$ and~$10$, as specified in the \ahref{x86-64.cfg}{\file{x86-64.cfg}}
configuration file.
We now try the following settings:
\begin{verbatim}
% sh run.sh -a 16 -s 10 -r 10000 > Y.01
% sh run.sh -a 16 -s 100 -r 1000 > Y.02
% sh run.sh -a 16 -s 1000 -r 100 > Y.03
% sh run.sh -a 16 -s 10000 -r 10 > Y.04
% sh run.sh -a 16 -s 100000 -r 1 > Y.05
\end{verbatim}
The additional \opt{-a 16}  command line option informs test executable
to use $16$ logical processors, hence running $8$ instances of
the ``\ltest{X}'' tests concurrently, as those tests all are two thread tests.
This technique of flooding the tested machine obviously
yields better ressource usage
and, according to our experience, favours outcome variability.

The following table summarises the results:
\begin{center}\let\handletest\xhandletest
\begin{tabular}{|l|l||r|r|r|r|r|}
\hline
& Y.01     & Y.02    & Y.03     & Y.04     & Y.05      \\
\hline
\hline
\handletest{X000} &  2.3k/800k& 602/800k& 465/800k & 551/800k & 297/800k  \\
\hline
\handletest{X001} &  2.9k/800k& 632/800k& 774/800k & 667/800k & 315/800k  \\
\hline
\handletest{X002} &  633/800k & 55/800k & 5/800k   & 7/800k   & 0/800k    \\
\hline
\handletest{X003} &  1.2k/800k& 182/800k& 152/800k & 390/800k & 57/800k   \\
\hline
\handletest{X004} &  2.4k/800k& 974/800k& 1.5k/800k& 2.4k/800k& 1.6k/800k \\
\hline
\handletest{X005} &  239/800k & 21/800k & 8/800k   & 0/800k   & 1/800k    \\
\hline
\handletest{X006} &  912/800k & 129/800k& 102/800k & 143/800k & 14/800k   \\
\hline
\end{tabular}


\end{center}
Again, we observe all targeted relaxed outcomes. In fact,
x86 relaxations are relatively easy to observe on our $16$
logical core machine.

Another test statistic of interest is
\emph{efficiency}, that is the number of targeted outcomes observed per
second:
\begin{center}\let\handletest\xhandletest
\begin{tabular}{|l||r|r|r|r|r|}
\hline
& Y.01& Y.02& Y.03& Y.04& Y.05 \\
\hline
\hline
\handletest{X000}& 285 & 2.2k& 6.6k& 9.2k& 4.2k \\
\hline
\handletest{X001}& 366 & 2.4k& 13k & 11k & 5.2k \\
\hline
\handletest{X002}& 78  & 212 & 71  & 140 &      \\
\hline
\handletest{X003}& 150 & 650 & 2.5k& 7.8k& 950  \\
\hline
\handletest{X004}& 288 & 3.7k& 25k & 59k & 32k  \\
\hline
\handletest{X005}& 28  & 72  & 114 &     & 17   \\
\hline
\handletest{X006}& 118 & 461 & 1.7k& 2.9k& 280  \\
\hline
\end{tabular}


\end{center}
As we can see, although the setting \opt{-s 10 -r 10000} yields the most
relaxed outcomes, it may not be considered as the most efficient.
Moreover, we see that tests \ltest{X002} and~\ltest{X005}
look more challenging than others.

Finally, it may be interesting to classify the ``\texttt{X}'' tests:
\begin{verbatim}
% mcycles @all | classify -arch X86
R
  X003 -> R+po+rfi-po : PodWW Wse Rfi PodRR Fre
  X006 -> R : PodWW Wse PodWR Fre
SB
  X000 -> SB+rfi-pos : Rfi PodRR Fre Rfi PodRR Fre
  X001 -> SB+rfi-po+po : Rfi PodRR Fre PodWR Fre
  X002 -> SB+mfence+rfi-po : MFencedWR Fre Rfi PodRR Fre
  X004 -> SB : PodWR Fre PodWR Fre
  X005 -> SB+mfence+po : MFencedWR Fre PodWR Fre
\end{verbatim}
One sees that two thread non-SC tests for x86 are basically of two kinds.




\section{Cross compiling, \label{affinity:experiment}affinity experiment}
In this section we describe how to produce the C~sources of tests
on a machine, while running the tests on another.
We also describe a sophisticated affinity experiment.

We assume a directory \ahref{tst-ppc}{\texttt{tst-ppc}}%
\ifhevea{} (\ahref{tst-ppc.tar}{archive})\fi,
that contains a series of litmus tests
and an index file~\ahref{tst-ppc/@all}{\file{@all}}.
Those tests where produced by the \prog{diycross}
tool. They illustrate variations of the
classical \ahref{tst-ppc/IRIW.litmus}{\ltest{IRIW}} test.
\ifhevea\begin{center}\img{IRIW}\end{center}\fi
More specifically, the \ltest{IRIW} variations are produced as follows
(see also Sec.~\ref{diycross:intro}):
\begin{verbatim}
% mkdir tst-ppc
% diycross -name IRIW -o tst-ppc Rfe PodRR,DpAddrdR,LwSyncdRR,EieiodRR,SyncdRR Fre Rfe PodRR,DpAddrdR,LwSyncdRR,EieiodRR,SyncdRR Fre
Generator produced 15 tests
\end{verbatim}


We target a Power7 machine described by the configuration file
\ahref{power7.cfg}{\file{power7.cfg}}:
\verbatiminput{power7.cfg}
One may notice the SMT (\emph{Simultaneaous Multi-Threading}) specification:
$4$-ways SMT (\verb+smt=4+), logical processors pertaining
to the same core being numbered in sequence (\verb+smt_mode = seq+) ---
that is, logical processors from the first core are $0$, $1$ ,$2$ and~$3$;
logical processors from the second core are $4$, $5$ ,$6$ and~$7$; etc.
The SMT specification is necessary to enable
custom affinity mode
(see Sec.~\ref{affinity:custom}).

One may also notice the specification of $0$ available logical processors
(\verb+avail=0+).
As affinity support is enabled (\verb+affinity=incr0+),
test executables will find themselves
the number of logical processors available on the target machine.


We compile tests to C-sources packed in archive \file{a.tar}
and upload the archive to the target power7 machine as follows:
\begin{verbatim}
% litmus -mach power7 -o a.tar tst-ppc/@all
% scp a.tar power7:
\end{verbatim}
Then, on \texttt{power7} we unpack the archive and produce executable tests
as follows:
\begin{verbatim}
power7% tar xmf a.tar
power7% make -j 8
gcc -D_GNU_SOURCE -Wall -std=gnu99 -O -m64 -pthread -O2 -c affinity.c
gcc -D_GNU_SOURCE -Wall -std=gnu99 -O -m64 -pthread -O2 -c outs.c
gcc -D_GNU_SOURCE -Wall -std=gnu99 -O -m64 -pthread -S IRIW+eieios.c
...
\end{verbatim}

As a starter, we can check the effect of available logical processor detection
and custom affinity control (option \opt{+ca})
by passing the command line option \opt{-v} to one test executable,
for instance
\texttt{IRIW.exe}:
\begin{verbatim}
power7% ./IRIW.exe -v +ca
./IRIW.exe -v +ca
IRIW: n=8, r=10, s=1000, st=1, +ca, p='0,1,2,3,4,5,6,7,8,9,10,11,12,13,14,15,16,17,18,19,20,21,22,23,24,25,26,27,28,29,30,31'
thread allocation: 
[23,22,3,2] {5,5,0,0}
[7,6,15,14] {1,1,3,3}
[11,10,5,4] {2,2,1,1}
[21,20,27,26] {5,5,6,6}
[9,8,25,24] {2,2,6,6}
[31,30,13,12] {7,7,3,3}
[19,18,29,28] {4,4,7,7}
[1,0,17,16] {0,0,4,4}
...
\end{verbatim}
We see that our machine \texttt{power7} features $32$ logical processors
numbered from $0$ to~$31$
(cf \verb+p=...+ above) and will thus run \verb+n=8+ concurrent
instances of the $4$~thread IRIW~test.
Additionally allocation of threads to logical processors is shown:
here, the four threads of the test are partitioned into two groups, which are
scheduled to run on different cores. For example, threads $0$ and~$1$ of
the first instance of the test will run on logical processors $23$ and~$22$
(core~$5$); while threads $2$ and~$3$ will run on logical
processors $3$ and~$2$ (core~$0$).

Our experiment consists in running all tests
with affinity increment (see Sec.~\ref{defi}) being from $0$
and then $1$ to~$8$ (option \opt{-i $i$}),
as well as in random and custom affinity mode
(options \opt{+ra} and~\opt{+ca}):
\begin{verbatim}
power7% for i in $(seq 0 8)
> do
> sh run.sh -i $i > Z.0$i
> done
power7% sh run.sh +ra > Z.0R
power7% sh run.sh +ca > Z.0C

\end{verbatim}
The following table summarises the results, with X meaning that the targeted
relaxed outcome is observed:
\begin{center}\def\tstdir{tst-ppc}\let\handletest\xhandletest
\begin{tabular}{|l||c|c|c|c|c|c|c|c|c|c|c|}
\hline
& Z.00& Z.01& Z.02& Z.03& Z.04& Z.05& Z.06& Z.07& Z.08& Z.0C& Z.0R \\
\hline
\hline
\handletest{IRIW}& X &     & X & X &     & X & X & X & X & X & X  \\
\hline
\handletest{IRIW+addr+po}& X &     & X & X &     &     &     &     & X  & X &      \\
\hline
\handletest{IRIW+addrs}&     &     & X &     &     &     &     &     &     & X & X   \\
\hline
\handletest{IRIW+eieio+addr}&     &     & X & X &     &     &     &     &     & X &      \\
\hline
\handletest{IRIW+eieio+po}&     &     & X & X &     &     &     &     &     & X &      \\
\hline
\handletest{IRIW+eieios}&     &     & X & X &     &     &     &     &     & X & X  \\
\hline
\handletest{IRIW+lwsync+addr}&     &     & X & X &     &     &     &     &     & X &      \\
\hline
\handletest{IRIW+lwsync+eieio}&     &     & X & X &     &     &     &     &     & X &      \\
\hline
\handletest{IRIW+lwsync+po}& X  &     & X & X &     &     &     & X &     & X &      \\
\hline
\handletest{IRIW+lwsyncs}&     &     & X &     &     &     &     &     &     &  X &      \\
\hline
\handletest{IRIW+sync+addr}&     &     & X &     &     &     &     &     &     & X &      \\
\hline
\handletest{IRIW+sync+eieio}&     &     & X &     &     &     &     &     &     & X &      \\
\hline
\handletest{IRIW+sync+lwsync}&     &     & X &     &     &     &     &     &     & X &      \\
\hline
\handletest{IRIW+sync+po}& X &     & X & X & X &     &     &     &     & X & X  \\
\hline
\handletest{IRIW+syncs}&     &     &     &     &     &     &     &     &     &     &      \\
\hline
\end{tabular}


\end{center}
On sees that all possible relaxed outcomes shows up with proper affinity
control. More precisely, setting the affinity increment to $2$ or resorting
to custom affinity result in the same effect:
the first two threads of the test run on one core, while the last two threads
of the test run on a different core.
As demonstrated by the experiment, this allocation of test threads to cores
suffices to favour relaxed outcomes for all tests except for
\ahref{tst-ppc/IRIW+syncs.litmus}{\ltest{IRIW+syncs}},
where the \texttt{sync} fences forbid them.


\section{Cross running, testing low-end devices}
Together \litmus{} options \ahrefloc{gcc}{\opt{-gcc}}
and~\ahrefloc{linkopt}{\opt{-linkopt}} permit using
a C~cross compiler. For instance, assume that \prog{litmus}
runs on machine~$A$ and that \opt{crossgcc}, a cross compiler for machine~$C$,
is available on machine~$B$. Then, the following sequence of
commands can be used to test machine~$C$:
\begin{verbatim}
A% litmus -gcc crossgcc -linkopt -static -o C-files.tar ...
A% scp C-files.tar B:

B% tar xf C-files.tar
B% make
B% tar cf /tmp/C-compiled.tar .
B% scp /tmp/C-compiled.tar C:

C% tar xf C-compiled.tar
C% sh run.sh
\end{verbatim}
Alternatively, using option \ahrefloc{crossrun}{\opt{-crossrun $C$}},
one can avoid copying the archive \verb+C-compiled.tar+ to machine~$C$:
\begin{verbatim}
A% litmus -crossrun C -gcc crossgcc -linkopt -static -o C-files.tar ...
A% scp C-files.tar B:

B% tar xf C-files.tar
B% make
B% sh run.sh
\end{verbatim}
More specifically, option \opt{-crossrun $C$} instructs the \file{run.sh}
script to upload executables individually to machine~$C$, just before running
them. Notice that executables are removed from~$C$ once run.

We illustrate the crossrun feature by testing \ltest{LB} variations on
an ARM-based Tegra3 ($4$ cores) tablet.
Test \ahref{tst-arm/LB.litmus}{\ltest{LB}} (load-buffering) exercises
the following ``causality'' loop:
\begin{center}\cycle{LB}\end{center}
That is, thread~0 reads the values stored to location~\texttt{x} by thread~1,
thread~1 reads the values stored to location~\texttt{y} by thread~0,
and both threads read ``before'' they write.

We shall consider tests with varying interpretations of ``before'':
the write may simply follow the read in program order
(\texttt{po} in test names),
may depend on the read (\texttt{data} and \texttt{addr}), or
they may be some fence in-betweeen
(\texttt{isb} and \texttt{dmb}).
We first generate tests \ahref{tst-arm}{\texttt{tst-arm}}%
\ifhevea{} (\ahref{tst-arm.tar}{archive}) \fi
with \prog{diycross}:
\begin{verbatim}
% mkdir tst-arm
% diycross -arch ARM -name LB -o tst-arm PodRW,DpDatadW,DpCtrldW,ISBdRW,DMBdRW Rfe PodRW,DpDatadW,DpCtrldW,ISBdRW,DMBdRW Rfe
Generator produced 15 tests
\end{verbatim}

We use the following, \afile{tegra3.cfg}, configuration file:
\verbatiminput{tegra3.cfg}
Notice the ``cross-compilation'' section:
the name of the gcc cross-compiler is \texttt{arm-linux-gnueabi-gcc},
while the adequate version of the target ARM variant
and static linking are specified.

We compile the tests from litmus source files to C~source files in
directory \texttt{TST} as follows:
\begin{verbatim}
% mkdir TST
% litmus -mach tegra3 -crossrun app_81@wifi-auth-188153:2222 tst-arm/@all -o TST
\end{verbatim}
The extra option \texttt{-crossrun app\_81@wifi-auth-188153:2222}
specifies the address to log onto the tablet by \texttt{ssh},
which is connected on a local WiFi network and runs a \texttt{ssh} daemon
that listens on port~$2222$.

We compile to executables and run them  as as follows:
\begin{verbatim}
% cd TST
% make
arm-linux-gnueabi-gcc -Wall -std=gnu99 -march=armv7-a -O2 -pthread -O2 -c outs.c
arm-linux-gnueabi-gcc -Wall -std=gnu99 -march=armv7-a -O2 -pthread -O2 -c utils.c
arm-linux-gnueabi-gcc -Wall -std=gnu99 -march=armv7-a -O2 -pthread -S LB.c
...
% sh run.sh > ARM-LB.log
\end{verbatim}
\ifhevea(Complete \ahref{ARM-LB.log}{run log}.) \fi
It is important to notice that the shell script \texttt{run.sh} runs
on the local machine, not on the remote tablet.
Each test executable is copied (by using \texttt{scp}) to the tablet, runs there
and is deleted (by using \texttt{ssh}), as can be seen with \texttt{sh}
``\texttt{-x}'' option:
\begin{verbatim}
% sh -x run.sh 2>&1 >ARM-LB.log | grep -e scp -e ssh
+ scp -P 2222 -q ./LB.exe app_81@wifi-auth-188153:
+ ssh -p 2222 -q -n app_81@wifi-auth-188153 ./LB.exe -q  && rm ./LB.exe
+ scp -P 2222 -q ./LB+data+po.exe app_81@wifi-auth-188153:
+ ssh -p 2222 -q -n app_81@wifi-auth-188153 ./LB+data+po.exe -q  && rm ./LB+data+po.exe
...
\end{verbatim}


Experiment results can be extracted from the log file quite easily,
by reading the ``Observation'' information from test output:
\begin{verbatim}
% grep Observation ARM-LB.log
Observation LB Sometimes 1395 1998605
Observation LB+data+po Sometimes 360 1999640
Observation LB+ctrl+po Sometimes 645 1999355
Observation LB+isb+po Sometimes 1676 1998324
Observation LB+dmb+po Sometimes 18 1999982
Observation LB+datas Never 0 2000000
Observation LB+ctrl+data Never 0 2000000
Observation LB+isb+data Sometimes 654 1999346
Observation LB+dmb+data Never 0 2000000
Observation LB+ctrls Never 0 2000000
Observation LB+isb+ctrl Sometimes 1143 1998857
Observation LB+dmb+ctrl Never 0 2000000
Observation LB+isbs Sometimes 2169 1997831
Observation LB+dmb+isb Sometimes 178 1999822
Observation LB+dmbs Never 0 2000000
\end{verbatim}
What is observed (\texttt{Sometimes}) or not (\texttt{Never}) is the occurence
of the non-SC behaviour of tests. All tests have the same structure
and the observation of the non-SC behaviour can be interpreted as
some read not being ``before'' the write by the same thread.
This situation occurs for plain program order (plain test \ltest{LB} and
\texttt{po} variations) and for the \texttt{isb} fence.

\ifhevea
The following graph summarises the observations and illustrates
that data dependencies, control dependencies and the \texttt{dmb} barrier
apparently suffice to restore SC in the case of the LB family.
\begin{center}\img{LB-kinds}\end{center}
In the graph above, a red node means an observation of the non-SC behaviour.
\fi

\section{Finding\label{example:invalid} and showing invalid executions}
We now describe a complete experiment that will use some of
the additional tools we distribute.
The experiment aims at comparing ARM machines with the uniproc model.
The uniproc model is a very relaxed memory model that only enforces
single-thread correctness --- See Sec.~\ref{defuniproc}.
Single thread correctness can be defined as accesses to same
location by the same thread do not contradict communication candidate
relaxations --- See Sec.~\ref{communication:cr}.

\subsection*{Test generation}
We first produce a few simple tests that access one memory location
only. We use the \prog{diy} test generator with configuration
file~\ahref{tst-co/CO.conf}{\texttt{CO.conf}}:
\verbatiminput{tst-co/src/CO.conf}
The first section above describes generated cycles: the vocabulary
of candidate relaxations (we also consider DMB barriers),
the size of cycles, the (maximal) number of threads, and a strong limitation
on the number of consecutive internal candidate relaxations
(the setting \opt{ins 2} by rejecting 2 or more consecutive internal candidate
relaxations in effect forbids sequences of internal candidate
relaxations).
We also specify \prog{diy} \opt{uni} mode (See Sec.~\ref{uni:def})
that will allow test generation from one-location cycles,
and replace final conditions by the observation of test outcomes
(\opt{cond observe}).
Here we go:
\begin{verbatim}
% diy -conf CO.conf
Generator produced 31 tests
% ls
2+2W+dmb+pos.litmus  MP+dmbs.litmus     SB+dmb+pos.litmus  W+RR+dmb.litmus
2+2W+dmbs.litmus     MP+pos+dmb.litmus  SB+dmbs.litmus     W+RR.litmus
2+2W+poss.litmus     MP+poss.litmus     SB+poss.litmus     W+RW+dmb.litmus
@all                 R+dmb+pos.litmus   S+dmb+pos.litmus   W+RW.litmus
CO.conf              R+dmbs.litmus      S+dmbs.litmus      WW+dmb.litmus
LB+dmb+pos.litmus    R+pos+dmb.litmus   S+pos+dmb.litmus   WW.litmus
LB+dmbs.litmus       R+poss.litmus      S+poss.litmus
LB+poss.litmus       RW+dmb.litmus      WR+dmb.litmus
MP+dmb+pos.litmus    RW.litmus          WR.litmus
\end{verbatim}
\input{BASIC.tex}\newcommand{\basic}[1]{\ahref{BASIC-#1.html}{\ltest{#1}}}%
We get 31~tests, 
available in directory \ahref{tst-co}{\texttt{tst-co}}%
\ifhevea{} (\ahref{tst-co.tar}{archive})\fi.
Amongst those $31$~tests, are five tests that may exhibit
the five ``basic'' uniproc violations.
Those five basic uniproc violations are three direct contraditions of
program order and communication candidate relaxations Ws,Rf and~Fr ---
which we show
as executions of the tests \basic{WW}, \basic{RW} and~\basic{WR}:
\begin{center}
\img{WW}\qquad\qquad
\img{RW}\qquad\qquad
\img{WR}
\end{center}
plus two contradictions of
program order and communication candidate relaxation
sequences Ws;Rf and~Fr;Rf,
which we show
as executions of the tests \basic{W+RW} and~\basic{W+RR}:
\begin{center}
\img{W+RW}\qquad
\img{W+RR}
\end{center}

Namely, due to the transitivity of Ws and to the definition of Fr
(that implies $\textrm{Fr;Ws} \subseteq \textrm{Fr}$) all sequences of
communications are covered by the above listed five cases.

\input{MPPOSS.tex}\newcommand{\mpposs}[1]{\ahref{MPPOSS-#1.html}{#1}}%
However, having more tests than the five basic ones is relevant to
hardware testing, as the \textsc{uniproc} check will be exercised in
more contexts. For instance the test \mpposs{MP+poss}
may reveal $12$~different violations of \textsc{uniproc}.
We shall also test similar violations
in the presence of one or two \texttt{dmb} fences
(tests \mpposs{MP+dmb+pos}, \mpposs{MP+pos+dmb} and~\mpposs{MP+dmbs}).

\subsection{Running tests on hardware}

\subsubsection{Trimslice computer (a machine that runs linux)}
We first run our test set on a
\ahref{http://utilite-computer.com/web/trim-slice}{Trimslice computer},
which is powered by a NVIDIA cortex-A9 based Tegra2 chipset.
As our machine runs some linux distribution, we proceed by ordinary
cross-compilation (\emph{i.e.} we compile litmus tests into C sources
on our local machine and compile C files on the remote machine):
\begin{verbatim}
% litmus -mach trimslice -mem direct -st 1 -o /tmp/A.tar tst-co/@all
% scp /tmp/A.tar trimslice:/tmp/A.tar
\end{verbatim}
We use the \afile{trimslice.cfg} (\prog{litmus}) configuration file present
in the \prog{litmus} distribution, additionnally specifying
\ahrefloc{defmemorymode}{memory direct mode}
and a \ahrefloc{defstride}{stride value} of~\opt{1}.
We then copy the \texttt{A.tar} archive to our Trimslice machine.

We unpack and compile the C~sources on the remote Trimslice computer:
\begin{verbatim}
$ mkdir TST
$ cd TST
$ tar xf /tmp/A.tar
$ make -j 4 all
gcc -D_GNU_SOURCE -Wall -std=gnu99 -mcpu=cortex-a9 -marm -O2 -pthread -O2 -c affinity.c
gcc -D_GNU_SOURCE -Wall -std=gnu99 -mcpu=cortex-a9 -marm -O2 -pthread -O2 -c outs.c
gcc -D_GNU_SOURCE -Wall -std=gnu99 -mcpu=cortex-a9 -marm -O2 -pthread -O2 -c utils.c
gcc -D_GNU_SOURCE -Wall -std=gnu99 -mcpu=cortex-a9 -marm -O2 -pthread -S R+dmbs.c
...
gcc -D_GNU_SOURCE -Wall -std=gnu99 -mcpu=cortex-a9 -marm -O2 -pthread  -o W+RW.exe affinity.o outs.o utils.o W+RW.s
gcc -D_GNU_SOURCE -Wall -std=gnu99 -mcpu=cortex-a9 -marm -O2 -pthread  -o W+RR.exe affinity.o outs.o utils.o W+RR.s
\end{verbatim}
We are now ready for running the tests, we perform $10$~runs of the tests,
with varying strides, using a shell script \afile{trimslice.sh}.
\verbatiminput{trimslice.sh}
Notice that we also use the
\ahrefloc{affinity:runopt}{affinity setting \opt{-i 1}}.
We do so in order to accelerate the tests, as very little
can ne expected from a dual-core system as regards thread placement on cores.
We run the tests:
\begin{verbatim}
$ sh ./trimslice.sh
\end{verbatim}
After a bit less then $20$~minutes we get ten files
\texttt{CO.00} to~\texttt{CO.09} that we transfer back to our main machine,
in some sub-directory \texttt{trimslice}.

\subsubsection{\label{driverc:example}APQ8060 (a development board that runs Android)}
We then run our test set on a development board
powered by a Qualcomm \ahref{http://en.wikipedia.org/wiki/Snapdragon_(system_on_chip)#Snapdragon_S3}{APQ8060 system-on-chip} (dual-core Scorpion).
This board runs Android and is connected to our local computer
by the \emph{Android Debug Bridge} (adb).
We perform complete cross-compilation, \emph{i.e.} we shall compile
both litmus sources into C~sources and C~sources into executables
on our local machine. We first run \prog{litmus} as follows:
\begin{verbatim}
% mkdir -p SRC/DRAGON
% litmus -mach dragon -mem direct -st 1 -o SRC/DRAGON -driver C tst-co/@all
\end{verbatim}
We use the \afile{dragon.cfg} (\prog{litmus}) configuration file present
in the \prog{litmus} distribution:
\verbatiminput{dragon.cfg}
The configuration file defines the number of available cores
(\verb+avail = 2+),
gives some defaults for standard tests parameters,
specifies a few affinity settings
(\verb+affinity = incr0+ and \verb+force_affinity = true+), and then
define the C~cross-compiler and its options.
Notice that the C~sources are dumped into the pre-existing
directory (\opt{-o SRC/DRAGON}) and that we build a single executable
(\opt{-driver C}).
We then compile C~sources, on the local machine.
\begin{verbatim}
% cd SRC/DRAGON
% make -j 8
arm-linux-gnueabi-gcc -D_GNU_SOURCE -Wall -std=gnu99 -march=armv7-a -mthumb -O2 -pthread -O2 -c affinity.c
arm-linux-gnueabi-gcc -D_GNU_SOURCE -Wall -std=gnu99 -march=armv7-a -mthumb -O2 -pthread -O2 -c outs.c
arm-linux-gnueabi-gcc -D_GNU_SOURCE -Wall -std=gnu99 -march=armv7-a -mthumb -O2 -pthread -O2 -c utils.c
arm-linux-gnueabi-gcc -DASS -D_GNU_SOURCE -Wall -std=gnu99 -march=armv7-a -mthumb -O2 -pthread -S R+dmbs.c
...
arm-linux-gnueabi-gcc  -D_GNU_SOURCE -Wall -std=gnu99 -march=armv7-a -mthumb -O2 -pthread -static -o run.exe affinity.o outs.o utils.o R+dmbs.o MP+dmbs.o R+dmb+pos.o MP+dmb+pos.o WW+dmb.o 2+2W+dmbs.o S+dmbs.o 2+2W+dmb+pos.o S+dmb+pos.o WR+dmb.o SB+dmbs.o SB+dmb+pos.o R+pos+dmb.o RW+dmb.o LB+dmbs.o S+pos+dmb.o LB+dmb+pos.o W+RW+dmb.o MP+pos+dmb.o W+RR+dmb.o R+poss.o MP+poss.o WW.o 2+2W+poss.o S+poss.o WR.o SB+poss.o RW.o LB+poss.o W+RW.o W+RR.o run.o
rm W+R
...
\end{verbatim}




